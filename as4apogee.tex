%\documentclass{emulateapj}
\documentclass[11pt,preprint]{aastex}
%\documentclass{emulateapj}
\usepackage{apjfonts}
\usepackage{natbib}
\usepackage{graphicx}

%% Common Abbreviations %%
\newcommand{\um}{\mu{\rm m}}
\newcommand{\hst}{{\it HST\,\,}}
\newcommand{\ergs}{erg s$^{-1}$}
\newcommand{\etal}{{\rm et al.}}
\newcommand{\eg}{{\rm e.g.}}
\newcommand{\cf}{{\rm cf.}}
\newcommand{\ie}{{\rm i.e.}}
\newcommand{\etc}{{\rm etc.}}
\newcommand{\kms}{km s$^{-1}$}
\newcommand{\kgcm}{kg cm$^{-3}$}
\newcommand{\kgm}{kg m$^{-3}$}
\newcommand{\dr}{$\Delta R$}
\newcommand{\teff}{\ensuremath{T_{\mathrm{eff}}}}
\newcommand{\msol}{\ensuremath{\ \mathrm{M_{\odot}}}}
\newcommand{\vlos}{\ensuremath{v_{\mathrm{los}}}}
\newcommand{\feh}{\ensuremath{[\mathrm{Fe/H}]}}
\newcommand{\s}[2]{$\mathcal{S}_{#1,#2}$}
\newcommand{\jz}{\ensuremath{j_z}}
\newcommand{\jcirc}{\ensuremath{j_\mathrm{circ}}}
\newcommand{\zmed}{\ensuremath{z_\mathrm{med}}}
\newcommand{\cir}{\ensuremath{\epsilon}}
\newcommand{\jcrit}{\ensuremath{\epsilon_\mathrm{crit}}}
\newcommand{\Etot}{\ensuremath{E_\mathrm{tot}}}
\newcommand{\tform}{\ensuremath{t_\mathrm{form}}}
\newcommand{\rgc}{\ensuremath{r_\mathrm{gc}}}
\newcommand{\gcr}{\ensuremath{R_\mathrm{g}}}
\newcommand{\dgcr}{\ensuremath{\Delta R_\mathrm{g}}}
\newcommand{\rform}{\ensuremath{r_\mathrm{form}}}
\newcommand{\tcosmic}{\ensuremath{t_\mathrm{cosmic}}}
\newcommand{\vzdisp}{\ensuremath{\sigma_{\mathrm{v_z}}}}
\newcommand{\vrdisp}{\ensuremath{\sigma_{\mathrm{v_r}}}}
\newcommand{\vznon}{\ensuremath{\sigma_{\mathrm{v}_{z,non}}}}
\newcommand{\vzom}{\ensuremath{\sigma_{\mathrm{v}_{z,om}}}}
\def\kmsmpc{\,{\rm km\,s^{-1}\,Mpc^{-1}}}
\def\Mo{{\rm M_\odot}}
\def\LCDM{$\Lambda$CDM}
%% END %%

\shorttitle{After SDSS-IV White Paper}
\shortauthors{Bird \etal}

\begin{document}

\title{APOGEE-3: Anchoring Galactic Cosmology}

\author{Jonathan C. Bird \altaffilmark{1,2}, David W. Hogg, Melissa Ness, H.W. Rix, Gail Zasowski, Laura Inno, Andrew Casey [order pending]}

\altaffiltext{1}{Department of Physics and Astronomy, Vanderbilt University, 6301 Stevenson Center, Nashville, TN, 37235}
\altaffiltext{2}{VIDA Postdoctoral Fellow}

\begin{abstract}
We propose to use the current APOGEE spectrograph to do some awesome things. APOGEE-3 will be the first? spectroscopic survey of ~1E6 stars. We got your distance indicators, galactic cosmology, other side of the bulge, current state of the ISM from Cepheids, time domain programs including Cepheids and planets (and more), seeing the "disk wobble" in 3D, data-driven RV determination (for finding planets!), GAIA targeting info to give us whatever distance distribution we want, a global understanding of the MW today, .... we got it all!
\end{abstract}

\section{Overview}
The Galactic suveys of the past and coming decade are currently revolutionizing our understanding of both stellar astrophysics and the current configuration of our home Galaxy. By the end of SDSS-IV, the astronomical community will have produced high resolution spectra of over [N] Milky Way stars; imaged our Galaxy down to [N] mag; and the GAIA satellite will have measured pallaxes for [N] stars and space motions for [N] stars.[Galah, GAIA-ESO, SEGUE, A1 and A2, RAVE] Still, efforts to derive the formation history of our Galaxy from these data are hampered by 1) imprecise distances for stars that trace the considerable mass distributions of the midplane and inner Galaxy and 2) limited access to the present configuration of the Galaxy and not its past. In conjunction with the non-linear dynamics of disk evolution, these limitations suggest that the detailed "snapshot" of the Milky Way provided by Galactic surveys is more limited in its power to discriminate amongst different galaxy formation models than its proposed extent. To address both needs, we propose a near-Infrared survey of X Red Giants, $\sim 1\times10^6$ Red Clump stars, and $>50\%$ of all Galactic Cepheids. Precise distance measures to both RC stars and Cepheids are well-established while the young age of Cepheids makes them an ideal stellar tracer of the conditions of the ISM, establishing a robust boundary condition on the evolution of the Galaxy that will better connect where stars are currently observed to where they were born. Such a survey can be accomplished inexpensively with the current APOGEE spectrograph and without the need to move from the APO 2.5 meter. The perceived weakness of this instrument-telescope pairing after SDSS-IV is acutally a strength as APOGEE will be the only NIR Multi-object spectograph capable of observing bright Cepheids throughout the Galaxy. [Some statement about opportunity]






APOGEE itself has played an integral role, measuring individual abdunances and high precision RVs for X stars (by 2020). New statistical techniques (Cannon, data-driven), insights in to stellar astrophysics, and the powerful addition of astroseismic data has enabled us to further exploit stellar spectra to extract radii, mass, and evolutionary state.

Despite the major success: cartography, mapping bulge, mdf and kinematics. These results have not "solved" major issues facing Galaxy formation. We are closer than ever but we still don't have answers. This is because Galaxy formation models are simply differential equations... boundary conditions yada yada yada.

Our ability to answer these questions is hampered by our lack of knowledge about the early Galaxy: can't do anything about it ... and our lack of precision about the PRESENT of the Galaxy. 

Uncertainty in Migration, Maps are blurry etc. 

RCs and Cepheids can solve this.

We only have MOONs as competition and we can use their bright limit  against them. We won't be able to beat them going deep and faint, but we CAN get the bright cepheids and some RCs too.

Constrains migration (Binney paper, Rix plot)  etc.


Laura's stuff... How to  find the targets before then. Boom!




represent a transformational period of our is is truly the golden era of Galactic
%----------------------------
\section{Introduction}
\label{sec:intro}
%----------------------------
Spectrograph still amazing. Combined with targeting info from GAIA + IR capabilities + Cannon = BOOM. 

%----------------------------
\section{Cepheids}
\label{sec:cepheids}
%----------------------------

%----------------------------
\section{Red Clump Stars}
\label{sec:rcs}
%----------------------------

%----------------------------
\section{Requiremnets}
\label{sec:results}
%----------------------------
\subsection{R \& D}
- Data driven RVs
- Hardware already done. 
- Targeting changes for shorter exposures and higher cadence.

\subsection{Financial}
- COSTS: mostly people and facility operations

\subsection{People}
Who is going to lead this if it gets picked up?
%----------------------------

\section{Summary}
\label{sec:summary}
%----------------------------

\acknowledgements
We acknowledge All.

%\bibliographystyle{apj}
%\bibliography{refs}
%
\end{document}

