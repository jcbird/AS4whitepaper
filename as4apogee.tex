%\documentclass{emulateapj}
\documentclass[11pt,preprint]{aastex}
%\documentclass{emulateapj}
\usepackage{apjfonts}
\usepackage{natbib}
\usepackage{graphicx}

%% Common Abbreviations %%
\newcommand{\um}{\mu{\rm m}}
\newcommand{\hst}{{\it HST\,\,}}
\newcommand{\ergs}{erg s$^{-1}$}
\newcommand{\etal}{{\rm et al.}}
\newcommand{\eg}{{\rm e.g.}}
\newcommand{\cf}{{\rm cf.}}
\newcommand{\ie}{{\rm i.e.}}
\newcommand{\etc}{{\rm etc.}}
\newcommand{\kms}{km s$^{-1}$}
\newcommand{\kgcm}{kg cm$^{-3}$}
\newcommand{\kgm}{kg m$^{-3}$}
\newcommand{\dr}{$\Delta R$}
\newcommand{\teff}{\ensuremath{T_{\mathrm{eff}}}}
\newcommand{\msol}{\ensuremath{\ \mathrm{M_{\odot}}}}
\newcommand{\vlos}{\ensuremath{v_{\mathrm{los}}}}
\newcommand{\feh}{\ensuremath{[\mathrm{Fe/H}]}}
\newcommand{\s}[2]{$\mathcal{S}_{#1,#2}$}
\newcommand{\jz}{\ensuremath{j_z}}
\newcommand{\jcirc}{\ensuremath{j_\mathrm{circ}}}
\newcommand{\zmed}{\ensuremath{z_\mathrm{med}}}
\newcommand{\cir}{\ensuremath{\epsilon}}
\newcommand{\jcrit}{\ensuremath{\epsilon_\mathrm{crit}}}
\newcommand{\Etot}{\ensuremath{E_\mathrm{tot}}}
\newcommand{\tform}{\ensuremath{t_\mathrm{form}}}
\newcommand{\rgc}{\ensuremath{r_\mathrm{gc}}}
\newcommand{\gcr}{\ensuremath{R_\mathrm{g}}}
\newcommand{\dgcr}{\ensuremath{\Delta R_\mathrm{g}}}
\newcommand{\rform}{\ensuremath{r_\mathrm{form}}}
\newcommand{\tcosmic}{\ensuremath{t_\mathrm{cosmic}}}
\newcommand{\vzdisp}{\ensuremath{\sigma_{\mathrm{v_z}}}}
\newcommand{\vrdisp}{\ensuremath{\sigma_{\mathrm{v_r}}}}
\newcommand{\vznon}{\ensuremath{\sigma_{\mathrm{v}_{z,non}}}}
\newcommand{\vzom}{\ensuremath{\sigma_{\mathrm{v}_{z,om}}}}
\def\kmsmpc{\,{\rm km\,s^{-1}\,Mpc^{-1}}}
\def\Mo{{\rm M_\odot}}
\def\LCDM{$\Lambda$CDM}
%% END %%

\shorttitle{After SDSS-IV White Paper}
\shortauthors{Bird \etal}

\begin{document}

\title{APOGEE-3: Anchoring Galactic Cosmology}

\author{Jonathan C. Bird \altaffilmark{1,2}, David W. Hogg, Melissa Ness, H.W. Rix}

\altaffiltext{1}{Department of Physics and Astronomy, Vanderbilt University, 6301 Stevenson Center, Nashville, TN, 37235}
\altaffiltext{2}{VIDA Postdoctoral Fellow}

\begin{abstract}
We propose to use the current APOGEE spectrograph to do some awesome things. APOGEE-3 will be the first? spectroscopic survey of ~1E6 stars. We got your distance indicators, galactic cosmology, other side of the bulge, current state of the ISM from Cepheids, time domain programs including Cepheids and planets (and more), seeing the "disk wobble" in 3D, data-driven RV determination (for finding planets!), GAIA targeting info to give us whatever distance distribution we want, a global understanding of the MW today, .... we got it all!
\end{abstract}

%----------------------------
\section{Introduction}
\label{sec:intro}
%----------------------------
Spectrograph still amazing. Combined with targeting info from GAIA + IR capabilities + Cannon = BOOM. 


%----------------------------
\section{Requiremnets}
\label{sec:results}
%----------------------------
\subsection{R \& D}
- Data driven RVs
- Hardware already done. 
- Targeting changes for shorter exposures and higher cadence.

\subsection{Financial}
- COSTS: mostly people and facility operations

\subsection{People}
Who is going to lead this if it gets picked up?
%----------------------------

\section{Summary}
\label{sec:summary}
%----------------------------

\acknowledgements
We acknowledge All.

%\bibliographystyle{apj}
%\bibliography{refs}
%
\end{document}

